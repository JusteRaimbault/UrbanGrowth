\documentclass[11pt]{article}

\begin{document}
	
	
\title{City growth regularities}
	
	
The growth of cities, when represented by a single quantity which is generally population, present some striking regularities. These are at the first order independent from geography but also city size itself, as postulated by Gibrat's law proposing random growth with an uniform expected growth rate. This chapter reviews this model and associated random growth models such as Simon's model, considering them in an unified framework. We then describe simulation models proposed to explain deviations to Gibrat's law, including interaction models for systems of cities. A second part of this chapter is dedicated to the empirical test of growth regularities across main systems of cities worldwide. We finally test the sensitivity of these to the definition of cities, using the Global Human Settlement layer database. These patterns have implications for sustainable territorial planning at large scales, for example by suggesting leverages to reduce inequalities within systems of cities.



\cite{gabaix2004evolution}
\cite{batty2008size}
\cite{mitzenmacher2004brief}
\cite{gabaix1999zipf}



	
	
\bibliographystyle{apalike}
\bibliography{biblio}

	
\end{document}
