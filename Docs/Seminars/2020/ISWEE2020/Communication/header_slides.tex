
\documentclass{beamer}
\usetheme{ucl}

\usepackage[utf8]{inputenc}


%%% Increase the height of the banner: the argument is a scale factor >=1.0
%\setbeamertemplate{banner}[ucl][10.0]

%%% Change the colour of the main banner
%%% The background should be one of the UCL colours (except pink or white):
%%%   black,darkpurple,darkred,darkblue,darkgreen,darkbrown,richred,midred,
%%%   navyblue,midgreen,darkgrey,orange,brightblue,brightgreen,lightgrey,
%%%   lightpurple,yellow,lightblue,lightgreen,stone
\setbeamercolor{banner}{bg=darkpurple}
%\setbeamercolor{banner}{bg=yellow,fg=black}

%%% Add a stripe behind the banner
%\setbeamercolor{banner stripe}{bg=darkpurple,fg=black}

%%% The main structural elements
\setbeamercolor{structure}{fg=black}

%%% Author/Title/Date and slide number in the footline
\setbeamertemplate{footline}[author title date]

%%% Puts the section/subsection in the headline
% \setbeamertemplate{headline}[section]

%%% Puts a navigation bar on top of the banner
%%% For this to work correctly, the each \section command needs to be
%%% followed by a \subsection. Requires one extra compile.
% \setbeamertemplate{headline}[miniframes]
%%% Accepts an optional argument determining the width
% \setbeamertemplate{headline}[miniframes][0.3\paperwidth]


%%% Puts the frame title in the banner
%%% Won't work correctly with the above headline templates
%\useoutertheme{ucltitlebanner}
%%% Similar to above, but smaller (and puts subtitle on same line as title)
\useoutertheme[small]{ucltitlebanner}

%%% Gives block elements (theorems, examples) a border
% \useinnertheme{blockborder}
%%% Sets the body of block elements to be clear
% \setbeamercolor{block body}{bg=white,fg=black}

%%% Include CSML logo on title slide
%\titlegraphic{\includegraphics[width=0.16\paperwidth]{csml_logo}}

%%% Include CSML logo in bottom right corner of all slides
%\logo{\includegraphics[width=0.12\paperwidth]{csml_logo}}

%%% Set a background colour
% \setbeamercolor{background canvas}{bg=lightgrey}

%%% Set a background image
%%% Some sample images are available from the UCL image store:
%%%   https://www.imagestore.ucl.ac.uk/home/start
% \setbeamertemplate{background canvas}{%
%   \includegraphics[width=\paperwidth]{imagename}}



%%%%%% Some other settings that can make things look nicer
%%% Set a smaller indent for description environment
\setbeamersize{description width=2em}
%%% Remove nav symbols (and shift any logo down to corner)
\setbeamertemplate{navigation symbols}{\vspace{-2ex}}








\DeclareMathOperator{\Cov}{Cov}
\DeclareMathOperator{\Var}{Var}
\DeclareMathOperator{\E}{\mathbb{E}}
\DeclareMathOperator{\Proba}{\mathbb{P}}

\newcommand{\Covb}[2]{\ensuremath{\Cov\!\left[#1,#2\right]}}
\newcommand{\Eb}[1]{\ensuremath{\E\!\left[#1\right]}}
\newcommand{\Pb}[1]{\ensuremath{\Proba\!\left[#1\right]}}
\newcommand{\Varb}[1]{\ensuremath{\Var\!\left[#1\right]}}

% norm
\newcommand{\norm}[1]{\| #1 \|}

\newcommand{\indep}{\rotatebox[origin=c]{90}{$\models$}}





\usepackage{mathptmx,amsmath,amssymb,graphicx,bibentry,bbm,ragged2e}
\usepackage[english]{babel}

\makeatletter

\newcommand{\noun}[1]{\textsc{#1}}
\newcommand{\jitem}[1]{\item \begin{justify} #1 \end{justify} \vfill{}}
\newcommand{\sframe}[2]{\frame{\frametitle{#1} #2}}

\newenvironment{centercolumns}{\begin{columns}[c]}{\end{columns}}
%\newenvironment{jitem}{\begin{justify}\begin{itemize}}{\end{itemize}\end{justify}}



%\usetheme{Warsaw}
%\setbeamertemplate{footline}[text line]{}
%\setbeamertemplate{headline}{}
%\setbeamercolor{structure}{fg=purple!50!blue, bg=purple!50!blue}

%\setbeamersize{text margin left=15pt,text margin right=15pt}

%\setbeamercovered{transparent}


\@ifundefined{showcaptionsetup}{}{%
 \PassOptionsToPackage{caption=false}{subfig}}
\usepackage{subfig}

\usepackage[utf8]{inputenc}
\usepackage[T1]{fontenc}

\usepackage{multirow}


\makeatother

\def \draft {1}

\usepackage{xparse}
\usepackage{ifthen}
\DeclareDocumentCommand{\comment}{m o o o o}
{\ifthenelse{\draft=1}{
    \textcolor{red}{\textbf{C : }#1}
    \IfValueT{#2}{\textcolor{blue}{\textbf{A1 : }#2}}
    \IfValueT{#3}{\textcolor{ForestGreen}{\textbf{A2 : }#3}}
    \IfValueT{#4}{\textcolor{red!50!blue}{\textbf{A3 : }#4}}
    \IfValueT{#5}{\textcolor{Aquamarine}{\textbf{A4 : }#5}}
 }{}
}
\newcommand{\todo}[1]{
\ifthenelse{\draft=1}{\textcolor{red!50!blue}{\textbf{TODO : \textit{#1}}}}{}
}


