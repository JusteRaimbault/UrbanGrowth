%%%%%%%%%%%%%%%%%%%%%%% file template.tex %%%%%%%%%%%%%%%%%%%%%%%%%
%
% This is a general template file for the LaTeX package SVJour3
% for Springer journals.          Springer Heidelberg 2010/09/16
%
% Copy it to a new file with a new name and use it as the basis
% for your article. Delete % signs as needed.
%
% This template includes a few options for different layouts and
% content for various journals. Please consult a previous issue of
% your journal as needed.
%
%%%%%%%%%%%%%%%%%%%%%%%%%%%%%%%%%%%%%%%%%%%%%%%%%%%%%%%%%%%%%%%%%%%
%
% First comes an example EPS file -- just ignore it and
% proceed on the \documentclass line
% your LaTeX will extract the file if required
\begin{filecontents*}{example.eps}
%!PS-Adobe-3.0 EPSF-3.0
%%BoundingBox: 19 19 221 221
%%CreationDate: Mon Sep 29 1997
%%Creator: programmed by hand (JK)
%%EndComments
gsave
newpath
  20 20 moveto
  20 220 lineto
  220 220 lineto
  220 20 lineto
closepath
2 setlinewidth
gsave
  .4 setgray fill
grestore
stroke
grestore
\end{filecontents*}
%
\RequirePackage{fix-cm}
%
%\documentclass{svjour3}                     % onecolumn (standard format)
%\documentclass[smallcondensed]{svjour3}     % onecolumn (ditto)
\documentclass[smallextended]{svjour3}       % onecolumn (second format)
%\documentclass[twocolumn]{svjour3}          % twocolumn
%
\smartqed  % flush right qed marks, e.g. at end of proof
%
\usepackage{graphicx}
%
% \usepackage{mathptmx}      % use Times fonts if available on your TeX system
%
% insert here the call for the packages your document requires
%\usepackage{latexsym}
% etc.
%
% please place your own definitions here and don't use \def but
% \newcommand{}{}
%
% Insert the name of "your journal" with
% \journalname{myjournal}
%
\begin{document}

\title{Integrating and validating urban simulation models}
\subtitle{}

%\titlerunning{Short form of title}        % if too long for running head

\author{Juste Raimbault}

%\authorrunning{Short form of author list} % if too long for running head

\institute{J. Raimbault \at
              Center for Advanced Spatial Analysis, University College London\\
              \email{j.raimbault@ucl.ac.uk}
}

%\date{Received: date / Accepted: date}
\date{}
% The correct dates will be entered by the editor


\maketitle

\begin{abstract}

\keywords{Urban simulation models \and Model coupling \and Model validation}
% \PACS{PACS code1 \and PACS code2 \and more}
% \subclass{MSC code1 \and MSC code2 \and more}
\end{abstract}


\section{Introduction}



The integration of environmental dimensions within simulation models for territorial systems from theoretical and quantitative geography is an other important implementation project. For example, \cite{viguie2014downscaling} use an urban growth model to evaluate the local impact of climate change, but does not make associated aspects endogenous, such as energy price or the energy efficiency of the urban structure. Negative externalities such as congestion and pollution, directly linked to emissions, have a feedback for example on the location of activities. More generally, systematic links between a precise description of urban structure and energy efficiency, in a dynamical and endogenous way, remains to be established. Similarly, numerous studies in ecology which establish the impact of anthropic habitat disturbances would particularly benefit from a coupling with urban growth models, for example for a better management of the interface between the city and nature within the new urban regimes that are urban mega-regions \cite{hall2006polycentric}.

At the macroscopic scale, several couplings between models of systems of cities and ecological models or from environmental science are possible and desirable. In terms of territorial planning at small scales, an estimation of the impact of flows from interactions between cities on traversed areas can for example yield compromises between the distribution of flows and their environmental impact, and establish recommendations to minimise the global impact of the urban system, with. the constraint of keeping a reasonable resilience and economic performance. In the same way, the questions of production, storage and distribution of energy are endogenous to territorial systems, and the dynamics of infrastructures and associated entities can. be integrated within models of systems of cities. This research direction can in fact be generalised to any type of resource, and the SimpopLocal model \cite{pumain2017simpoplocal} and also the Marius model family \cite{cottineau2015modular} have integrated these aspects without however making them endogenous nor making them a central component of models. This aspect is linked to the previous point, given the difficulty to endogenise energy within macroeconomic models.

This research project, inspired by sciences of complexity and a principally geographical viewpoint, proposes the construction of bridges, coupling, and interdisciplinary dialogues, i.e. the construction of integrated theories. A way to understand such approaches is described by the complex systems roadmap \cite{bourgine2009french}. It combines horizontal integration (fundamental transversal questions at the intersection of different types of complex systems) with vertical integration (multiple levels coupled within multi-scale models). 
%\cite{raimbault2017applied} has furthermore proposed that the integration of knowledge domains (defined as components of knowledge construction processes, including theories, models and empirical facts, but also data, tools and methods) as typical of complex approaches.
%Considering modeling as a crucial element of knowledge production, this project is thus situated within approaches in theoretical and quantitative geography \citep{cuyala2014analyse} and geosimulation \citep{benenson2004geosimulation} in the case of simulation models. 





\section{Horizontal integration: coupling urban models and dimensions}





\section{Vertical integration: constructing multi-scale models}


Finally, the construction of multi-scale models will be in itself a crucial aspect, since it is a dimension in itself of the integration expected for the theories and models constructed. Following \cite{rozenblat2018conclusion}, given the multiple levels of articulation and interdependency that systems of cities have reached, the management and planning must necessarily be multi-scalar in order to take into account the geographical particularities while ensuring a global consistence which yield limited inequalities between territories.

%A possible concrete entry for a first exploration of multi-scale models, already aforementioned, would be the coupling of the evolutionary theory of cities with scaling theories. The first focuses on particularities of territorial entities while the second aims at producing universal laws, and both provide consistent explanations for scaling laws. A coupling strategy would be, relying on consolidated and comparable dataset, to (i) determine modular decompositions of territorial systems and corresponding scales, and the quantification of the universality of laws within these sub-systems by the study of scaling laws within and between subsystems; (ii) modeling this multi-scale system by. coupling. models of urban growth, which would be validated by scaling properties.

Moreover, a considerable methodological work would be necessary to elaborate coupling techniques between scales (for example as the hybrid modeling coupling agent-based models with differential equations for an epidemiological model \cite{banos2015coupling}), to determine the relevance of levels to be included and avoid ``ontological dead-ends'' \cite{roth2006reconstruction}, and to determine the nature of retroactions between scales and their necessity.




\section{Model exploration and validation}


Working to integrate models and theories necessarily necessitates a fine understanding of the modeling process itself, but also of stylised dynamics produced by simulation models. In the case of geographical models, such a knowledge has for example been developed within the Geodivercity ERC project \cite{pumain2017urban}, which lead to the conception of new methods answering to specific questions coming from geographical concerns (for example necessity and sufficiency of processes in a multi-modeling context \cite{reuillon2015}, exploration. of the feasible space of model outputs \cite{10.1371/journal.pone.0138212}).
%More generally in social sciences, modeling and simulation driven by new practices including model coupling, the use of high-performance computing for model exploration, open science practices, could be at the origin of new types of integrated knowledge \cite{banos2013pour}.

Besides, the development of methods and tools to improve the extraction of knowledge from simulation models, and an epistemological investigation on modeling practices, are crucial within this direction which can thus be understood as a theoretical and methodological direction. Indeed, methods to validate simulation models mostly remain to be developed. Methods for the exploration, sensitivity analysis, and validation, are essential for robust application of models, but also yield a better complementarity with other types of approaches since they can establish when modeling is not relevant.



\section{Discussion: towards evidence-based multi-scalar sustainable territorial planning}





% BibTeX users please use one of
%\bibliographystyle{spbasic}      % basic style, author-year citations
%\bibliographystyle{spmpsci}      % mathematics and physical sciences
\bibliographystyle{spphys}       % APS-like style for physics
\bibliography{biblio.bib}   % name your BibTeX data base


\end{document}
% end of file template.tex


%
%% For one-column wide figures use
%\begin{figure}
%% Use the relevant command to insert your figure file.
%% For example, with the graphicx package use
%  \includegraphics{example.eps}
%% figure caption is below the figure
%\caption{Please write your figure caption here}
%\label{fig:1}       % Give a unique label
%\end{figure}
%%
%% For two-column wide figures use
%\begin{figure*}
%% Use the relevant command to insert your figure file.
%% For example, with the graphicx package use
%  \includegraphics[width=0.75\textwidth]{example.eps}
%% figure caption is below the figure
%\caption{Please write your figure caption here}
%\label{fig:2}       % Give a unique label
%\end{figure*}
%%
%% For tables use
%\begin{table}
%% table caption is above the table
%\caption{Please write your table caption here}
%\label{tab:1}       % Give a unique label
%% For LaTeX tables use
%\begin{tabular}{lll}
%\hline\noalign{\smallskip}
%first & second & third  \\
%\noalign{\smallskip}\hline\noalign{\smallskip}
%number & number & number \\
%number & number & number \\
%\noalign{\smallskip}\hline
%\end{tabular}
%\end{table}

